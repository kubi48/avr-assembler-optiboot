\chapter{Verschiedene USB zu Seriell Wandler mit Linux}

\section*{}
Die klassische serielle Schnittstelle wird heute immer weniger verwendet.
Bei älteren Rechnern findet man diese Schnittstelle nach RS232 Standard häufiger.
Die serielle Schnittstelle ist aber für die Programmierung eines Mikrocontrollers
ohnehin nicht sehr praktisch, da die verwendeten Spannungspegel (etwa -12V und +12V)
nicht direkt verwendet werden können. Diese Spannungspegel müßten erst
wieder mit Pegelwandlern auf die bei Mikrocontrollern gebräuchlichen +5V oder 3.3V
zurückgewandelt werden.
Praktischer sind für den Zweck der Programmierung die USB zu Seriell Wandler,
da diese neben dem benötigten Spannungspegel auch noch die +5V oder +3.3V für
die Versorgung des Mikrocontrollers zur Verfügung stellen.
Unter Linux werden diese USB-Geräte meistens problemlos erkannt.
Es kann aber bei den Zugriffsrechten hapern.
Meistens werden die Zugangsknoten wie /dev/ttyUSB1 der Gruppe dialout zugewiesen.
Dann sollte man selbst (user) dieser Gruppe angehören.
Mit dem Kommando ,,usermod -a -G dialout \$USER'' sollte man der
Gruppe dialout angehören.


\section{Der CH340G und der CP2102 Wandler }
Untersucht habe ich eine Platine mit der Aufschrift BTE13-005A,
auf der ein CH340G Wandler von QinHeng Elektronics untergebracht ist. Die Platine hat einen Minischalter
zum Umschalten der VCC Spannung auf 3.3V oder 5V und einen 12 MHz Quarz.
Auf der einen Seite befindet sich ein USB-A Stecker und auf der anderen Seite
eine 6-polige Pin-Leiste mit den Signalen GND, CTS, VCC, TXD, RXD und DTR.
In einem chinesischen Datenblatt zum Chip kann man Angaben zu den unterstützten Baudraten finden.
Auf der Platine befindet sich eine mit VCC verbundene LED und eine weitere LED.

Die andere Platine mit dem CP2102 Wandler von Silicon Laboratories Inc. hat keine Aufschrift auf der Platine. 
Wie bei der CH340G Platine befindet sich auf der einen Seite der USB-A Stecker und auf
der gegenüberliegenden Seite eine 6-polige Stiftleiste mit den Signalen
3.3V, GND, +5V, TXD, RXD und DTR. Auf den beiden anderen Seiten läßt sich jeweils
eine 4-polige Stiftleiste nachrüsten mit den Signalen DCD, D3R, RTS und CTS
sowie RST, R1, /SUS und SUS. Alle für den Bootloader benötigten Signale befinden sich aber
auch bei dieser Platine auf der bereits bestückten Stiftleiste. 
Die Reihenfolge der Belegung ist aber unterschiedlich.
Auf der Platine mit dem CP2102 Wandler befinden sich außer dem Wandlerchip sehr wenige Bauteile,
eine LED für RXD, TXD und Power ist aber vorhanden.

Mit meinem Linux-Mint 17.2 wurden beide Wandler ohne weiteres erkannt.
Mit dem Kommando lsusb sieht man in der Ausgabe:
\begin{verbatim}
Bus 002 Device 093: ID 10c4:ea60 Cygnal Integrated Products, Inc. CP210x UART Bridge / myAVR mySmartUSB light
Bus 002 Device 076: ID 0403:6001 Future Technology Devices International, Ltd FT232 Serial (UART) IC
\end{verbatim}
Natürlich sind die Angaben zum Bus abhängig vom Rechner und dem verwendeten USB-Port.

Die beim Einstecken automatisch erzeugten Device Namen kann man sofort nach dem Einstecken
durch das Kommando ,,dmesg | tail -20'' herausfinden. Beim folgenden Beispiel habe
ich beide Ausgaben zusammengefaßt.
\begin{verbatim}
usb 2-4.2: new full-speed USB device number 94 using ohci-pci
usb 2-4.2: New USB device found, idVendor=1a86, idProduct=7523
usb 2-4.2: New USB device strings: Mfr=0, Product=2, SerialNumber=0
usb 2-4.2: Product: USB2.0-Serial
ch341 2-4.2:1.0: ch341-uart converter detected
usb 2-4.2: ch341-uart converter now attached to ttyUSB1
usb 2-4.5: new full-speed USB device number 93 using ohci-pci
usb 2-4.5: New USB device found, idVendor=10c4, idProduct=ea60
usb 2-4.5: New USB device strings: Mfr=1, Product=2, SerialNumber=3
usb 2-4.5: Product: CP2102 USB to UART Bridge Controller
usb 2-4.5: Manufacturer: Silicon Labs
usb 2-4.5: SerialNumber: 0001
cp210x 2-4.5:1.0: cp210x converter detected
usb 2-4.5: reset full-speed USB device number 93 using ohci-pci
usb 2-4.5: cp210x converter now attached to ttyUSB2
\end{verbatim}

Nach diesen Angaben und eigenen Experimenten habe ich die Tabellen~\ref{tab:CH340baudl} und~\ref{tab:CH340baudh}erstellt. Weil sich mit dem neueren Betriebssystem Linux Mint 18.3 teilweise bessere Ergebnisse ergeben haben, habe ich die Meßergebnisse mit diesem System gemessen.

\begin{table}[H]
  \begin{center}
    \begin{tabular}{| c | c | c || c | c | c || c |}
    \hline
    CH340G     & CH340G & CH340G   &  CP2102   & CP2102 & CP2102    & AVR \\
    supported  & stty  & measured  & supported & stty   & measured  & UBBR  \\
    BaudRate   & speed & BaudRate  & BaudRate  & speed  & BaudRate  & @16MHz \\
    \hline
    \hline
         50    & 50    &   50.00   &   (50)    & Error  &           &       \\
    \hline
         75    & 75    &   75.18   &   (75)    & Error  &           &        \\
    \hline
        100    & Error &    -      &  (100)    & Error  &           &        \\
    \hline
        110    & 110   &   109.3   &  (120)    & Error  &           &        \\
    \hline
        134    & 134   &   133.4   &  (134)    & Error  &           &       \\
    \hline
        150    & 150   &   150.4   &  (150)    & Error  &           &        \\
    \hline
        300    & 300   &   300.7   &   300     &  300   &   300.7    &       \\
    \hline
        600    & 600   &   602.4   &   600     &  600   &   598.8   &  3332  \\
    \hline
        900    & Error &    -      &  (900)    &  Error &    -      &  2221 \\
    \hline
       1200    & 1200  &   1204.8  &   1200    &  1200  &   1198    &   832  \\
    \hline
       1800    & 1800  &   1801.6  &   1800    &  1800  &   1802    &   555 \\
    \hline
       2400    & 2400  &   2409.6  &   2400    &  2400  &   2410    &   416  \\
    \hline
       3600    & Error &   -       &  (3600)   &  Error &    -      &   277  \\
    \hline
      (4000)   & Error &   -       &   4000    &  Error &    -      &   249  \\
    \hline
       4800    & 4800  &   4808    &   4800    &  4800  &   4808    &   207  \\
    \hline
      (7200)   & Error &   -       &   7200    &  Error &    -      &   138  \\
    \hline
       9600    & 9600  &   9616    &   9600    &  9600  &   9616    &   207  \\
    \hline
      14400    & Error &   -       &  14400    & Error  &    -      &   138  \\
    \hline
     (16000)   & Error &   -       &  16000    & Error  &    -      &   124  \\
    \hline
      19200    & 19200  &  19232   &   19200   &  19200 &  19232    &   103  \\
    \hline
    \end{tabular}
  \end{center}
  \caption{geprüfte Baudraten des CH340 und CP2102 im unteren Baud-Bereich}
  \label{tab:CH340baudl}
\end{table}
 



\begin{table}[H]
  \begin{center}
    \begin{tabular}{| c | c | c || c | c | c || c |}
    \hline
    CH340G     & CH340G & CH340G    &  CP2102   & CP2102 & CP2102    & AVR \\
    supported  & stty   & measured  & supported & stty   & measured  & UBBR  \\
    BaudRate   & speed  & BaudRate  & BaudRate  & speed  & BaudRate  & @16MHz \\
    \hline
    \hline
      28800  &  Error   &   -       &  28800    & Error  &    -      &   68 \\
    \hline
      33600  &  Error   &   -       & (33600)   & Error  &    -      &   59  \\
    \hline
      38400  &  38400   &  38.464k  &  38400    & 38400  &  38.464k   &   51  \\
    \hline
     (51200) &  Error   &   -       &  51200    & Error  &    -      &   38,  0.16\%  \\
    \hline
      56000  &  Error   &   -       &  56000    & Error  &    -      &   35, -0.79\%  \\
    \hline
      57600  &  57600   &  57.8k    &  57600    & 57600  &  57.472k   &   34, -0.79\%  \\
    \hline
     (64000) &  Error   &   -       &  64000    & Error  &    -      &   30,  0.80\%  \\
    \hline
      76800  &  Error   &   -       &  76800    & Error  &    -      &   25, 0.16\%  \\
    \hline
     115200  &  115200  &  115.6k   &  115200   & 115200 & 114.96k   &   16, 2.12\%  \\
    \hline
     128000  &  Error   &   -       &  128000   & Error  &    -      &   15, -2.34\%  \\
    \hline
     153600  &  Error   &   -       &  153600   & Error  &    -      &   12, 0.16\%  \\
    \hline
     230400  &  230400  &  229.9k   &  230400   & 230400 &  229.9k   &   8, -3.54\%  \\
    \hline
    (250000)  &  Error   &   -       &  250000   & Error  &    -      &   7, 0.00\%  \\
    \hline
    (256000)  &  Error   &   -       &  256000   & Error  &    -      &   7, -2.34\%  \\
    \hline
     460800  &  460800  &  460.8k   &  460800   & 460800 & 458.7k    &   -, >5\%  \\
    \hline
    (500000) &  500000  &  500.0k   &  500000   & 500000 & 500.0k    &   3, 0.00\%  \\
    \hline
    (576000) &  576000  & \bf{543,4k} &  576000   & 576000 & 571.4k   &   -, >5\%  \\
    \hline
     921600  &  921600  & \bf{851.2k} & 921600 & 921600 & 921.6k    &   -, >5\%  \\
    \hline
    (1000000) & 1000000  &  1000k    & (1000000) & 1000000 & \bf{921.6k}  &   1, 0.00\%  \\
    \hline
    (1200000) &  Error   &   -       & (1200000) & Error   &   -       &  -, >5\%  \\
    \hline
    1500000  & 1500000  &  1498k    & (1500000) & 1500000 & 1498k   &  -, >5\%  \\
    \hline
    2000000  & 2000000  &  2000k    & (2000000) & 2000000 & \bf{1504k}    & 0, 0.00\%  \\
    \hline
   (3000000) & 3000000  &  3007k    & (3000000) & 2000000 &   -       &  -, >5\%   \\
    \hline
    \end{tabular}
  \end{center}
  \caption{geprüfte Baudraten des CH340 und CP2102 im oberen Baud-Bereich}
  \label{tab:CH340baudh}
\end{table}

Bei beiden Tabellen fällt auf, daß nicht alle von den Herstellern angegebenen Baudraten
mit dem Linux stty Kommando einstellbar sind. In den meisten Fällen wird ein Fehler gemeldet,
aber leider nicht immer. Beim CP2102 Kontroller wird die Fehlergrenze bei den
Baudraten 1 MBaud und 2 MBaud deutlich überschritten. Auch bei den 576 kBaud ist die
Abweichung etwas höher als erwartet. 
Bei der Messung der eingestellten Baudrate fällt nur die Einstellung 576000 und 921600 des CH340G
Kontrollers negativ auf. Die anderen Einstellungen bleiben bei der Toleranz unkritisch.
Wo die Ursachen für diese Auffälligkeiten liegen, habe ich nicht erörtert.
Bei der Dokumentation zum CH340G scheint aber nicht alles mit der gelieferten Version
übereinzustimmen. Die Einstellungen 500 kBaud und 1 MBaud sind zwar möglich, aber nicht in der
Dokumentation erwähnt.


\section{Der PL-2303 und der FT232R Wandler}
Diesmal habe ich eine Platine mit der Aufschrift ,,SBT5329'' und dem PL-2303RX Wandler von Profilic Technology
und eine Platine ,,FTDI Basic 1'' mit einem FT232RL Wandler von Future Technology Devices untersucht.

Die SBT5329 Platine hat auf der einen Seite einen USB-A Stecker und auf der anderen Seite eine 5-polige
Stiftleiste mit den Signalen +5V, GND, RXD, TXD und 3.3V. Außerdem findet man auf der Platine noch
einen 12 MHz Quarz und drei Leuchtdioden Tx, Rx und Power. Die Steuersignale der seriellen Schnittstelle
sind bei dieser Platine nicht auf Stiftleisten herausgeführt. 
Der PL-2303 Chip stellt aber die Handshake-Signale der seriellen Schnittstelle bereit.

Die FTDI Basic 1 Platine hat auf der einen Seite eine USB-B-Buchse und auf der anderen Seite eine
6-polige Steckerleiste mit den Signalen GND, CTS, 5V, TXD, RXD und DTR. Außer dem FT232RL Chip sind nur wenige weitere Bauteile auf der Platine zu finden. Zwei LED's für Tx und Rx sind jedenfalls auch vorhanden.

Wie auch bei den beiden anderen Platinen werden die Geräte unter Linux Mint 17.1 einwandfrei erkannt.
\begin{verbatim}
Bus 002 Device 095: ID 067b:2303 Prolific Technology, Inc. PL2303 Serial Port
Bus 002 Device 076: ID 0403:6001 Future Technology Devices International, Ltd FT232 Serial (UART) IC
\end{verbatim}
So zeigen sich die Geräte mit dem Kommando lsusb. Mit dem Kommando ,,dmesg | tail -20'' sieht man dann
unmittelbar nach dem Einstecken die Meldungen:
\begin{verbatim}
usb 2-4.5: new full-speed USB device number 95 using ohci-pci
usb 2-4.5: New USB device found, idVendor=067b, idProduct=2303
usb 2-4.5: New USB device strings: Mfr=1, Product=2, SerialNumber=0
usb 2-4.5: Product: USB-Serial Controller
usb 2-4.5: Manufacturer: Prolific Technology Inc.
pl2303 2-4.5:1.0: pl2303 converter detected
usb 2-4.5: pl2303 converter now attached to ttyUSB1
\end{verbatim}
beziehungsweise
\begin{verbatim}
usb 2-4.3: new full-speed USB device number 96 using ohci-pci
usb 2-4.3: New USB device found, idVendor=0403, idProduct=6001
usb 2-4.3: New USB device strings: Mfr=1, Product=2, SerialNumber=3
usb 2-4.3: Product: FT232R USB UART
usb 2-4.3: Manufacturer: FTDI
usb 2-4.3: SerialNumber: A50285BI
ftdi_sio 2-4.3:1.0: FTDI USB Serial Device converter detected
usb 2-4.3: Detected FT232RL
usb 2-4.3: Number of endpoints 2
usb 2-4.3: Endpoint 1 MaxPacketSize 64
usb 2-4.3: Endpoint 2 MaxPacketSize 64
usb 2-4.3: Setting MaxPacketSize 64
usb 2-4.3: FTDI USB Serial Device converter now attached to ttyUSB0
\end{verbatim}

In der Tabelle~\ref{tab:PL2302baudl} wird die Baudrate für den FT232R Kontroller
auch unter 300 Baud eingestellt, obwohl der Chip das nicht kann.
Aber das stty Kommando akzeptiert diese Einstellungen ohne eine Fehlermeldung.
Natürlich ist die resultierende Baudrate für diese Einstellungen falsch.

\begin{table}[H]
  \begin{center}
    \begin{tabular}{| c | c | c || c | c | c || c |}
    \hline
    PL2303     & PL2303 & PL2303   &  FT232R   & FT232R & FT232R    & AVR \\
    supported  & stty  & measured  & supported & stty   & measured  & UBBR  \\
    BaudRate   & speed & BaudRate  & BaudRate  & speed  & BaudRate  & @16MHz \\
    \hline
    \hline
         75    &  75   &  75.18    &  (75)     &   75   &  \bf{415}  &        \\
    \hline
      (110)    &  110  &  109.9    &  (110)    & 110    & \bf{278} &        \\
    \hline
      (134)    &  134  &  135.1    &  (134)    & 134    & \bf{502}   &       \\
    \hline
        150    & 150   &  149.8    &  (150)    & 150    & \bf{832} &        \\
    \hline
        300    & 300   &  300.7    &   300     & 300    & 300.72     &       \\
    \hline
        600    & 600   &  602.4    &   600     & 600    & 602.4     &  3332  \\
    \hline
      (900)    & Error &    -      &   900     & Error  &    -      &  2221 \\
    \hline
       1200    & 1200  &  1204.8   &   1200    & 1200   & 1212    &   832  \\
    \hline
       1800    & 1800  &  1801.6   &   1800    & 1800   & 1809.6    &   555 \\
    \hline
       2400    & 2400  &  2409.6   &   2400    & 2400   & 2424    &   416  \\
    \hline
       3600    & Error &   -       &   3600    & Error  &    -      &   277  \\
    \hline
       4800    & 4800  &  4808     &   4800    & 4800   & 4831      &   207  \\
    \hline
       7200    & Error &   -       &   7200    &  Error &    -      &   138  \\
    \hline
       9600    & 9600  &  9616     &   9600    &  9600  &  9664     &   207  \\
    \hline
      14400    & Error &   -       &  14400    & Error  &    -      &   138  \\
    \hline
      19200    & 19200  & 19232    &  19200    & 19200  & 19320     &   103  \\
    \hline
    \end{tabular}
  \end{center}
  \caption{geprüfte Baudraten des PL-2303 und FT232R im unteren Baud-Bereich}
  \label{tab:PL2302baudl}
\end{table}

In der Tabelle~\ref{tab:PL2302baudh} wird die Baudrate für den FT232R Konverter
richtig gesetzt, wenn das stty Kommando keinen Fehler meldet.
Ich kenne den Grund nicht, warum einige Baudraten für den FT232R Chip nicht vom
stty Kommando akzeptiert werden.
Nur die Baudrate 576000 könnte vom Chip besser eingestellt werden, als hier gemessen.

\begin{table}[H]
  \begin{center}
    \begin{tabular}{| c | c | c || c | c | c || c |}
    \hline
    PL2303     & PL2303 & PL2303   &  FT232R   & FT232R & FT232R    & AVR \\
    supported  & stty  & measured  & supported & stty   & measured  & UBBR  \\
    BaudRate   & speed & BaudRate  & BaudRate  & speed  & BaudRate  & @16MHz \\
    \hline
    \hline
    \hline
      28800  &  Error   &   -       &  28800    & Error  &    -      &   68 \\
    \hline
     (33600) &  Error   &   -       &  33600    & Error  &    -      &   59  \\
    \hline
      38400  &  38400   &  38.464k  &  38400    & 38400  &  38.6k   &   51  \\
    \hline
    (51200)  &  Error   &   -       &  51200    & Error  &    -      &   38,  0.16\%  \\
    \hline
    (56000)  &  Error   &   -       &  56000    & Error  &    -      &   35, -0.79\%  \\
    \hline
      57600  &  57600   &  57.8k    &  57600    & 57600  &  57.8k    &   34, -0.79\%  \\
    \hline
    (64000)  &  Error   &   -       &  64000    & Error  &    -      &   30,  0.80\%  \\
    \hline
    (76800)  &  Error   &   -       &  76800    & Error  &    -      &   25, 0.16\%  \\
    \hline
     115200  &  115200  &  115.6k   &  115200   & 115200 &  115.6k   &   16, 2.12\%  \\
    \hline
    (128000) &  Error   &   -       &  128000   & Error  &    -      &   15, -2.34\%  \\
    \hline
    (153600) &  Error   &   -       &  153600   & Error  &    -      &   12, 0.16\%  \\
    \hline
     230400  &  230400  &  231.2k   &  230400   & 230400 &  231.2k   &   8, -3.54\%  \\
    \hline
    (250000) &  Error   &   -       &  250000   & Error  &    -      &   7, 0.00\%  \\
    \hline
    (256000) &  Error   &   -       &  256000   & Error  &    -      &   7, -2.34\%  \\
    \hline
     460800  &  460800  &  460.8k   &  460800   & 460800  & 465.1k   &   -, >5\%  \\
    \hline
    (500000) &  500000  &  500.0k   &  500000   & 500000 & 500.0k    &   3, 0.00\%  \\
    \hline
    (576000) &  Error  &    -       &  576000 & 576000 & \bf{588.24k} &   -, >5\%  \\
    \hline
     921600  &  921600  &  925.6k   & 921600   & 921600 &  930.4k   &   -, >5\%  \\
    \hline
   (1000000) & 1000000 &   1000k    &  1000000 & 1000000 & 1005k    &   1, 0.00\%  \\
    \hline
   (1200000)  &  Error &   -        & 1200000  & Error   &   -      &  -, >5\%  \\
    \hline
   (1500000) & Error  &    1482k    & 1500000  & 1500000 &  1509k   &  -, >5\%  \\
    \hline
   (2000000) & 2000000 &   2010k   & 2000000 & 2000000 &  2020k   & 0, 0.00\%  \\
    \hline
   (3000000) & 3000000 &   3007k   & 3000000  & 3000000 &  3007k    &  -, >5\%   \\
    \hline
    \end{tabular}
  \end{center}
  \caption{geprüfte Baudraten des PL-2303 und FT232R im oberen Baud-Bereich}
  \label{tab:PL2302baudh}
\end{table}


\section{Der USB-serial Wandler mit der ATmega16X2 Software}

Auf manchen Arduino UNO Platinen wird ein ATmega16X2 als USB-seriell Wandler benutzt.
Deswegen möchte ich die Untersuchungen der einstellbaren Baudraten auch für diese Lösung
durchführen.
In der ersten Tabelle~\ref{tab:mega16xbaudl} für den unteren Baudraten-Bereich fällt nur
negativ auf, daß die Baudraten unter 600 Baud von stty ohne Fehlermeldung akzeptiert werden.
Ab 600 Baud aufwärts werden die Einstellungen ohne Fehlermeldung brauchbar eingehalten.

\begin{table}[H]
  \begin{center}
    \begin{tabular}{| c | c | c || c |}
    \hline
    Mega16X2   & Mega16X2 & Mega16X2   & AVR \\
    supported  & stty  & measured  & UBBR  \\
    BaudRate   & speed & BaudRate  & @16MHz \\
    \hline
    \hline
         75    &  75   & \bf{956} &         \\
    \hline
        110    &  110  & \bf{1120}  &         \\
    \hline
        134    &  134  & \bf{757.6} &        \\
    \hline
        150    & 150   & \bf{1914} &         \\
    \hline
        300    & 300   & \bf{778}  &        \\
    \hline
        600    & 600   &  599     &   3332  \\
    \hline
        900    & Error &   -      &   2221 \\
    \hline
       1200    & 1200  &  1198    &    832  \\
    \hline
       1800    & 1800  &  1802    &    555 \\
    \hline
       2400    & 2400  &  2395    &    416  \\
    \hline
       3600    & Error &    -     &    277  \\
    \hline
       4800    & 4800  &  4808    &    207  \\
    \hline
       7200    & Error &    -     &    138  \\
    \hline
       9600    & 9600  &   9616   &    207  \\
    \hline
      14400    & Error &    -     &    138  \\
    \hline
      19200    & 19200  &  19232  &    103  \\
    \hline
    \end{tabular}
  \end{center}
  \caption{geprüfte Baudraten des ATmega16X2 im unteren Baud-Bereich}
  \label{tab:mega16xbaudl}
\end{table}

Bei der Tablelle~\ref{tab:mega16xbaudh} für den oberen Baudbereich kann man
erkennen, daß auch nicht alle Zahlenwerte ohne Fehlermeldung richtig
eingestellt werden. 
Wünschenswert wäre natürlich eine bessere Rückmeldung von stty, welche Baudraten
für das angewählte Interface eingestellt werden können. 

\begin{table}[H]
  \begin{center}
    \begin{tabular}{| c | c | c || c |}
    \hline
    Mega16X2  & Mega16X2 & Mega16X2   & AVR \\
    supported  & stty  & measured  & UBBR  \\
    BaudRate   & speed & BaudRate  & @16MHz \\
    \hline
    \hline
    \hline
      28800  &  Error   &   -       &    68 \\
    \hline
      33600  &  Error   &   -       &    59  \\
    \hline
      38400  &  38400   &  38.312k  &    51  \\
    \hline
      51200  &  Error   &   -       &    38,  0.16\%  \\
    \hline
      56000  &  Error   &   -       &    35, -0.79\%  \\
    \hline
      57600  &  57600   &  58.82k    &    34, -0.79\%  \\
    \hline
      64000  &  Error   &   -       &    30,  0.80\%  \\
    \hline
      76800  &  Error   &   -       &    25, 0.16\%  \\
    \hline
     115200  &  115200  &  116.9k   &    16, 2.12\%  \\
    \hline
     128000  &  Error   &   -       &    15, -2.34\%  \\
    \hline
     153600  &  Error   &   -       &    12, 0.16\%  \\
    \hline
     230400  &  230400  &  221.0k   &    8, -3.54\%  \\
    \hline
     250000  &  Error   &   -       &    7, 0.00\%  \\
    \hline
     256000  &  Error   &   -       &    7, -2.34\%  \\
    \hline
    (460800) &  460800  & \bf{500.0k} &    -, >5\%  \\
    \hline
     500000  &  500000  &  500.0k   &    3, 0.00\%  \\
    \hline
    (576000)  & 576000 & \bf{667k}  &    -, >5\%  \\
    \hline
    (921600) &  921600  & \bf{995.0k} &    -, >5\%  \\
    \hline
   (1000000) & 1000000 &   1000k    &    1, 0.00\%  \\
    \hline
   (1200000) &  Error   &   -        &   -, >5\%  \\
    \hline
   (1500000) & Error  & \bf{2000k} &   -, >5\%  \\
    \hline
    2000000  & 2000000 &   2000k   & 0, 0.00\%  \\
    \hline
   (3000000) & 3000000 & \bf{2000k} &   -, >5\%   \\
    \hline
    \end{tabular}
  \end{center}
  \caption{geprüfte Baudraten des ATmega16X2 im oberen Baud-Bereich}
  \label{tab:mega16xbaudh}
\end{table}

\section{Der Pololu USB AVR Programmer v2.1}

Bei meinem Linux System werden bei eingestecktem Pololu Programmer zwei serielle
Schnittstellen erkannt, /dev/ttyACM0 und /dev/ttyACM1 .
Mit der ersten serielle Schnittstelle wird die ISP-Schnittstelle angesteuert.
Die zweite Schnittstelle /dev/ttyACM1 steht zur freien Verfügung.
Die Signale der seriellen Schnittstelle sind auf einer zusätzlichen
Buchsenleiste verfügbar (TX, RX, B=DTR, A=frei).

Die Ergebnisse meiner Untersuchungen der zusätzlichen seriellen Schnittstelle 
habe ich für den unteren Baud-Bereich
in der Tabelle~\ref{tab:pololubaudl} festgehalten.
Ab 300 Baud macht der Pololu ein gutes Bild, alle von stty akzeptierten
Baudraten werden sehr gut eingehalten.

\begin{table}[H]
  \begin{center}
    \begin{tabular}{| c | c | c || c |}
    \hline
    getestete  & stty  & gemessene  & AVR UBBR  \\
    BaudRate   & speed & BaudRate  & @16MHz \\
    \hline
    \hline
         75    &  75   & \bf{415} &         \\
    \hline
        110    &  110  & \bf{275}  &         \\
    \hline
        134    &  134  & \bf{500} &        \\
    \hline
        150    & 150   & \bf{830} &         \\
    \hline
        300    & 300   & 300      &        \\
    \hline
        600    & 600   &  600     &   3332  \\
    \hline
        900    & Error &   -      &   2221 \\
    \hline
       1200    & 1200  &  1200    &    832  \\
    \hline
       1800    & 1800  &  1800    &    555 \\
    \hline
       2400    & 2400  &  2400    &    416  \\
    \hline
       3600    & Error &    -     &    277  \\
    \hline
       4800    & 4800  &  4800    &    207  \\
    \hline
       7200    & Error &    -     &    138  \\
    \hline
       9600    & 9600  &   9600   &    207  \\
    \hline
      14400    & Error &    -     &    138  \\
    \hline
      19200    & 19200  &  19200  &    103  \\
    \hline
    \end{tabular}
  \end{center}
  \caption{geprüfte Baudraten der Pololu seriellen Schnittstelle im unteren Baud-Bereich}
  \label{tab:pololubaudl}
\end{table}


Bei den in der Tabelle~\ref{tab:pololubaudh} gezeigten Ergebnissen ist bei allen
von stty akzeptierten Baudraten nur 576000 unbrauchbar wegen zu hoher Abweichung.

\begin{table}[H]
  \begin{center}
    \begin{tabular}{| c | c | c || c |}
    \hline
    getestete  & stty  & gemessene  & AVR UBBR  \\
    BaudRate   & speed & BaudRate  & @16MHz \\
    \hline
    \hline
    \hline
      28800  &  Error   &   -       &    68 \\
    \hline
      33600  &  Error   &   -       &    59  \\
    \hline
      38400  &  38400   &  38.32k  &    51  \\
    \hline
      51200  &  Error   &   -       &    38,  0.16\%  \\
    \hline
      56000  &  Error   &   -       &    35, -0.79\%  \\
    \hline
      57600  &  57600   &  58.86k    &    34, -0.79\%  \\
    \hline
      64000  &  Error   &   -       &    30,  0.80\%  \\
    \hline
      76800  &  Error   &   -       &    25, 0.16\%  \\
    \hline
     115200  &  115200  &  115.28k   &    16, 2.12\%  \\
    \hline
     128000  &  Error   &   -       &    15, -2.34\%  \\
    \hline
     153600  &  Error   &   -       &    12, 0.16\%  \\
    \hline
     230400  &  230400  &  230.56k   &    8, -3.54\%  \\
    \hline
     250000  &  Error   &   -       &    7, 0.00\%  \\
    \hline
     256000  &  Error   &   -       &    7, -2.34\%  \\
    \hline
    (460800) &  460800  & 461.6k &    -, >5\%  \\
    \hline
     500000  &  500000  &  500.0k   &    3, 0.00\%  \\
    \hline
    (576000)  & 576000 & \bf{541.8k}  &    -, >5\%  \\
    \hline
    (921600) &  921600  & 923.9k &    -, >5\%  \\
    \hline
   (1000000) & 1000000 &   998.8k    &    1, 0.00\%  \\
    \hline
   (1200000) &  Error   &   -        &   -, >5\%  \\
    \hline
   (1500000) & 1500000  & 1501.2k &   -, >5\%  \\
    \hline
    2000000  & 2000000 &   2000k   & 0, 0.00\%  \\
    \hline
   (3000000) & 3000000 &  3000k &   -, >5\%   \\
    \hline
    \end{tabular}
  \end{center}
  \caption{geprüfte Baudraten der Pololu seriellen Schnittstelle im oberen Baud-Bereich}
  \label{tab:pololubaudh}
\end{table}

